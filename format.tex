% format.tex - 格式设置文件
\newcommand{\chineseabstract}{
    \noindent\zihao{-5}\songti\textbf{摘要:}\songti
}

\newcommand{\chinesekeywords}{
    \noindent\zihao{-5}\songti\textbf{关键词:}
}

\newcommand{\englishabstract}{
    \noindent\zihao{5}{\bfseries\fontspec{Times New Roman}[Weight=Bold]Abstract: }
}

\newcommand{\englishkeywords}{
    \noindent\zihao{5}{\bfseries\fontspec{Times New Roman}[Weight=Bold]Key words: }
}

% 节标题格式
\ctexset{section={format={\zihao{4}\heiti}}}
\ctexset{subsection={format={\zihao{5}\heiti}}}
\ctexset{subsubsection={format={\zihao{5}\songti}}}

% 图表标题格式
\DeclareCaptionFont{sfive}{\zihao{-5}}
\captionsetup[figure]{font=sfive,justification=centering}
\captionsetup[table]{font=sfive,justification=centering}

% 修改后的方案
\newcommand{\bottomnote}{
    \vfill
    \bottomnotefont
    \bottomnotespacing
    \noindent\rule{\linewidth}{0.4pt}\\
    \noindent 收稿日期:20**-**-**\\
    \noindent 基金项目:基金名称(编号)(六号字体):国家自然科学基金资助项目(00000001);\\
    \noindent 通信作者:李~三(1998——),男,教授,博士生导师,博士,研究方向为图像处理、数据挖掘等. E-mail: ****@shu.edu.cn (建议使用单位邮箱)
}